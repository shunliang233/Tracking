\chapter{Introduction}
\section{ATLAS Software Documentation}
\begin{itemize}
	\item FASER 的工作环境基本上是从 ATLAS 的环境修改来的,因此需要先了解 ATLAS 的工作环境。
	\item ATLAS Software Documentation 是关于 ATLAS 软件 Athena 的文档。
	\begin{itemize}
		\item Guides 栏目是为 Athena Developer 准备的,想要阅读 FASER 的径迹重建算法就需要查看这一栏。
		\item Analysis SW Tutorial 栏目是为数据分析人员准备的,介绍了 ATLAS 利用 Athena 程序包进行数据分析的流程。
		\item Other Tutorials 栏目是一些零碎的进阶项目。
	\end{itemize}
\end{itemize}

\section{Athena Framework}
\subsection{Project}
对 Athena 源代码应用不同的编译配置可以得到侧重于不同功能的软件,称为不同的 Project, 其中 Athena Project 具有最全面的功能。

\subsection{C++ 核心代码}
使用 C++ 编写核心代码 (Components):
\begin{itemize}
	\item Algorithm: 每一个使用 Athena 处理的 job 都由 algorithm chain 完成具体工作。
	\item Tool: 位于特定 Algorithm 中可以调用的小工具。
	\item Service: 每个 Algorithm 都可以调用的通用小工具。
\end{itemize}

\subsection{Python 接口代码}
使用 Python 编写接口代码,使用 component accumulator 指示整个 job 应该调用哪些 Algorithm, Tool, Service.

\section{使用 Athena}
在 \verb|/cvmfs| 上安装有现成的编译好的 Athena Project, 因此不需要自己从源代码编译,只需要 \verb|source| 一个 \verb|.sh| 文件即可使用 Athena 环境:

\verb|export ATLAS_LOCAL_ROOT_BASE=/cvmfs/atlas.cern.ch/repo/ATLASLocalRootBase
	  source ${ATLAS_LOCAL_ROOT_BASE}/user/atlasLocalSetup.sh
	  asetup Athena,main,latest
	  python -m AthExHelloWorld.HelloWorldConfig|

最后一行是运行测试模块,测试 Athena 环境是否被成功加载。





\chapter{Athena Configuration}
The configuration system of Athena allows us to assemble and customize components (algorithms, services and tools) for a workflow. It is done by creation and manipulation of python objects, and known as \verb|Component Accumulator| based configuration.

\section{Configuration System 的工作原理}
\begin{itemize}
	\item Each C++ component defines a set of configurable parameters (*),
	\begin{itemize}
		\item There is rich, yet restricted set of types of configurable parameters that are supported,
		\item Typically reasonable default values are defined as well.
	\end{itemize}
	\item During compilation of Athena Project, a database entry is made containing information about these properties.
	\item During the configuration this database is queried for information about a component and as a result python class is generated with class attributes corresponding to the properties defined in C++.
	\item Set of python script creates python objects and manipulates them (set properties). (*)
	\item Resulting set of python objects produce a serialized/textual representation of the configuration.
	\item The configuration is read in C++ program and populates the global dictionary with all settings.
	\item C++ components during the initialization fetch the values and set configurable parameters (see 1st point) accordingly.
\end{itemize}

The steps marked with (*) are places where developer intervention is needed. That is to define configurable parameters and to prepare scripts setting them. Other steps are automatized. FASER 开发的 calypso 就是在 Athena 的基础上定义了自己的 C++ components (algorithms, services and tools) 和用于设置这些 C++ components 的 Python 脚本。使用 CMake 编译 calypso 实际上就是编译其中的 C++ components, 编译完后才能在 Python 脚本中使用这些 components 来处理 job.





